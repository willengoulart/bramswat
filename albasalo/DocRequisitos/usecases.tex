\newcommand{\pgs}{\hspace{0.6cm}\=\hspace{0.5cm}\=\hspace{0.5cm}\=\hspace{0.5cm}\=\hspace{0.5cm}\=\hspace{0.5cm}\= \kill}

\subsection{Use Cases}

\subsubsection{Entra no sistema}
\begin{tabbing}
\pgs
\>\>O usuário se conecta ao sistema.\\
\>\>Insere seu login e senha.\\
\>\>Se forem válidos,\\
\>\>\>ele consegue entrar no sistema.\\
\>\>Caso contrário,\\
\>\>\>o sistema pede o login e a senha novamente.
\end{tabbing}

\subsubsection{Usuário adiciona Item}
\begin{tabbing}
\pgs
\>\>Usuário tem acesso ao Acervo.\\
\>\>Seleciona uma opção para adicionar um novo item.\\
\>\>Escolhe o tipo do item.\\
\>\>Completa os atributos.\\
\>\>Se forem inválidos (por exemplo, incompletos),\\
\>\>\>preenche de novo.\\
\>\>Caso contrário (válidos),\\
\>\>\>Se o item já existe,\\
\>\>\>\>o usuário é avisado e questionado se deseja adicionar outro item.\\
\>\>\>\>Caso a resposta seja sim,\\
\>\>\>\>\>Volta à adição de novo item.\\
\>\>\>\>Caso contrário,\\
\>\>\>\>\>Volta ao Acervo.\\
\>\>\>Senão (item novo),\\
\>\>\>\>O sistema requisita uma confirmação.\\
\>\>\>\>Se a resposta for confirma,\\
\>\>\>\>\>O item é adicionado ao acervo.\\
\>\>\>\>Senão,\\
\>\>\>\>\>Volta à adição de novo item.
\end{tabbing}

\subsubsection{Usuário adiciona Exemplar}
\begin{tabbing}
\pgs
\>\>Usuário tem acesso à sua Coleção.\\
\>\>Escolhe adicionar um novo exemplar.\\
\>\>Escolhe o tipo do exemplar.\\
\>\>Escolhe o item associado.\\
\>\>Se o item associado não está no sistema,\\
\>\>\>Pede-se que o usuário faça o cadastro.\\
\>\>\>Se o usuário não o fizer,\\
\>\>\>\>Volta à adição de exemplar.\\
\>\>Completa os atributos.\\
\>\>Se forem inválidos (por exemplo, incompletos),\\
\>\>\>Preenche de novo.\\
\>\>Caso contrário (válidos)\\
\>\>\>O sistema requere que se vincule uma negociação.\\
\>\>\>Se nenhuma é vinculada,\\
\>\>\>\>Volta à Coleção.\\
\>\>\>O sistema requisita uma confirmação.\\
\>\>\>Se a resposta for confirma,\\
\>\>\>\>O exemplar é adicionado à coleção com a negociação.\\
\>\>\>Senão,\\
\>\>\>\>Volta à adição de novo exemplar.
\end{tabbing}

\subsubsection{Usuário remove Exemplar}
\begin{tabbing}
\pgs
\>\>Usuário tem acesso à sua Coleção.\\
\>\>Escolhe remover um de seus exemplares.\\
\>\>O sistema pede uma confirmação.\\
\>\>Se confirma,\\
\>\>\>As negociações associadas ao exemplar são removidas,\\
\>\>\>O exemplar é removido.\\
\>\>Se não for confirmado,\\
\>\>\>Volta à sua Coleção.
\end{tabbing}

\subsubsection{Usuário remove Negociação}
\begin{tabbing}
\pgs
\>\>Usuário tem acesso às suas negociações.\\
\>\>Escolhe remover uma de suas negociações.\\
\>\>O sistema pede uma confirmação.\\
\>\>Se confirma,\\
\>\>\>A negociação é removida.\\
\>\>Se não for confirmado,\\
\>\>\>Volta às suas negociações.
\end{tabbing}

\subsubsection{Usuário busca por Item}
\begin{tabbing}
\pgs
\>\>Um usuário logado no sistema tem acesso a uma interface de busca.\\
\>\>O usuário deseja buscar por um Item.\\
\>\>O usuário refina sua busca selecionando uma opção de buscar por Item do Acervo.\\
\>\>O usuário preenche os ``campos'' da busca.\\
\>\>O sistema efetua as pesquisas.\\
\>\>Os resultados da pesquisa são exibidos para o usuário.
\end{tabbing}

\subsubsection{Usuário busca por Usuário}
\begin{tabbing}
\pgs
\>\>Um usuário logado no sistema tem acesso a uma interface de busca.\\
\>\>O usuário deseja buscar por um Usuário.\\
\>\>O usuário refina sua busca selecionando uma opção de buscar por Usuário do Sistema.\\
\>\>O usuário preenche os ``campos'' da busca.\\
\>\>O sistema efetua as pesquisas.\\
\>\>Os resultados da pesquisa são exibidos para o usuário.
\end{tabbing}

\subsubsection{Usuário visualiza o Perfil de um Usuário}
\begin{tabbing}
\pgs
\>\>Um usuário logado no sistema deseja visualizar o perfil de um usuário.\\
\>\>O usuário dá uma entrada que sinaliza ao sistema para mostrar o perfil de um
usuário (clicar\\
num botão ou algo parecido).\\
\>\>O perfil do usuário é exibido na tela.
\end{tabbing}

\subsubsection{Usuário denuncia conteúdo impróprio}
\begin{tabbing}
\pgs
\>\>O usuário logado, ao navegar pelo sistema, nota o cadastro de um item
(ou exemplar) que, em\\
sua opinião, tem um conteúdo impróprio.\\
\>\>Ele marca-o como conteúdo impróprio.\\
\>\>O sistema requisita uma confirmação.\\
\>\>\>Caso o usuário confirme,\\
\>\>\>\>o sistema envia uma notificação (e-mail) ao administrador sobre a ocorrência.\\
\>\>\>\>O administrador do sistema decidirá o que será feito.\\
\>\>\>Caso contrário,\\
\>\>\>\>volta à tela em que o usuário estava.
\end{tabbing}

\subsubsection{Usuário pede ajuda}
\begin{tabbing}
\pgs
\>\>O usuário, já logado no sistema, tem uma dúvida sobre o funcionamento do sistema.\\
\>\>O usuário vai para a área de ajuda do sistema.\\
\>\>O usuário envia sua dúvida ao sistema através do próprio canal de comunicação deste.\\
\>\>O sistema envia uma notificação (e-mail) ao administrador com a dúvida.\\
\>\>O administrador responde ao usuário através do canal de comunicação.
\end{tabbing}

\subsubsection{Administrador adiciona Item}
\begin{tabbing}
\pgs
\>\>O administrador está dentro do sistema.\\
\>\>Seleciona uma opção para adicionar um novo item dentro da tela do acervo.\\
\>\>Na tela de adição de item:\\
\>\>\>Escolhe o tipo do item,\\
\>\>\>Completa atributos:\\
\>\>\>\>Se eles não forem válidos, preenche de novo.\\
\>\>\>\>Senão,\\
\>\>\>\>\>Se o item já existe, o administrador é avisado e questionado.\\
\>\>\>\>\>Se deseja adicionar outro item:\\
\>\>\>\>\>\>Caso sim, volta à adição de item.\\
\>\>\>\>\>\>Caso contrário, volta ao acervo.\\
\>\>\>\>\>Se é realmente novo, o sistema requisita uma confirmação.\\
\>\>\>\>\>\>Caso seja confirmada, o item é adicionado ao acervo.\\
\>\>\>\>\>\>Caso contrário volta à adição de novo exemplar.\\
\end{tabbing}

\subsubsection{Administrador remove (qualquer) Item}
\begin{tabbing}
\pgs
\>\>O administrador tem acesso a todos os itens.\\
\>\>Pode remover qualquer um.\\
\>\>Ao ser removido, todos os exemplares desse item serão removidos.\\
\>\>Será possível selecionar vários de uma vez para remoção múltipla.
\end{tabbing}

\subsubsection{Administrador remove (qualquer) Exemplar}
\begin{tabbing}
\pgs
\>\>O administrador tem acesso a todos os exemplares.\\
\>\>Pode remover qualquer um.\\
\>\>Ao ser removido, o exemplar não estará mais no sistema.\\
\>\>Será possível selecionar vários de uma vez para remoção múltipla.
\end{tabbing}

\subsubsection{Administrador remove (qualquer) Negociação}
\begin{tabbing}
\pgs
\>\>O administrador tem acesso a todos os exemplares e pode remover qualquer um.\\
\>\>Ao apagar, a negociação não estará mais ativa, nem para acesso futuro.\\
\>\>Será possível selecionar vários de uma vez para remoção múltipla.
\end{tabbing}

\subsubsection{Administrador remove (qualquer) Usuário}
\begin{tabbing}
\pgs
\>\>O administrador tem acesso a todos os usuários.\\
\>\>Pode remover qualquer um.\\
\>\>\>Todos os exemplares ligados a esse usuário serão removidos.\\
\>\>Ao remover um usuário, esse estará fora do sistema e não poderá fazer login.\\
\>\>Será possível selecionar vários de uma vez para remoção múltipla.
\end{tabbing}
